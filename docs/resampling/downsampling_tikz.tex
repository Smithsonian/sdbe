 %% cartoon explaining down-sampling
 % http://tex.stackexchange.com/questions/124878/declare-function-for-tikzpicture
\begin{figure}
\begin{tikzpicture}[
declare function={ 
    funcx(\x) = -0.1*sin(2*pi*2.8*\x r) + 0.2*sin(2*pi*1.3*\x r) + 
                0.3*sin((2*pi*0.9*\x - 0.3) r) + 0.6*sin(2*pi*0.5*\x r);
    funchx(\x) = 0.3*sin((2*pi*0.9*\x - 0.3) r) + 0.6*sin(2*pi*0.5*\x r);},
]
  % frequency domain 

  %% |X(f)|
  \draw [<->, rounded corners, very thick](0,1.5) -- (0,0) --(6.5,0) node[below] {$f$};
  \node[left] at (0,0.7) {$|X|$};
  \foreach \i in {1,...,6}
    \draw[thick] (\i, 0) -- (\i,0.1);
  \node[below] at (3,0) {$f_0/2$};
  \draw[very thick] (0,1) to [out=10, in=170] (0.5,1.03)
                        to [out=350, in=190] (2,0.95) 
                        to [out=10, in=170] (2.5,1)
                        to [out=-10, in=110] (3,0);
  \draw[very thick,dashed] (6-0,1) to 
                            [out=170, in=10] (6-0.5,1.03) to 
                            [out=190, in=350] (6-2,0.95) to 
                            [out=170, in=10] (6-2.5,1) to [out=190, in=70] (6-3,0);

  %% |H(f)|
  \draw [<->, rounded corners, very thick](0,-0.5) -- (0,-2) -- (6.5,-2);
  \node[left] at (0,0.7-2) {$|H|$};
  \foreach \i in {1,...,6}
    \draw[thick] (\i, -2) -- (\i,0.1-2);
  \node[below] at (1,0-2) {$f_0/2M$}; \node[below] at (6,0-2) {$f_0$};
  \draw[very thick] (0,1-2) -- (0.8,1-2) -- (1,0-2);
  \draw[very thick,dashed] (5,0-2) -- (5.2,1-2) -- (6,1-2);

  %% |W(f)|
  \draw [<->, rounded corners, very thick](0,-2.5) -- (0,-4) -- (6.5,-4);
  \node[left] at (0,0.7-4) {$|HX|$};
  \foreach \i in {1,...,6}
    \draw[thick] (\i, -4) -- (\i,0.1-4);
  \node[below] at (1,0-4) {$f_0/2M$}; \node[below] at (6,0-4) {$f_0$};
  \draw[very thick] (0,1-4) to [out=10, in=170] (0.5,1.03-4) to [out=350, in=105] (1,0-4);
  \draw[very thick,dashed] (6-0,1-4) to [out=170, in=10] (6-0.5,1.03-4) to [out=190, in=75] (6-1,0-4);

  %% Y(f)
  \draw [<->, rounded corners, very thick](0,-4.5) -- (0,-6) -- (6.5,-6) node[below] {$f$};
  \node[left] at (0,0.7-6) {$|Y|$};
  \foreach \i in {1,...,6}
    \draw[thick] (\i, -6) -- (\i,0.1-6);
  \node[below] at (1,0-6) {$f_1/2$}; 
  \draw[very thick] (0,1-6) to [out=10, in=170] (0.5,1.03-6) to [out=350, in=105] (1,0-6);
  \draw[very thick,dashed] (1,0-6) to [out=75, in=190]  (1.5,1.03-6) to [out=10, in=170] 
                            (2, 1-6) to [out=10,in=170] (2.5, 1.03-6) to [out=350, in=105] (3,-6);
  \draw[very thick,dashed] (3,0-6) to [out=75, in=190]  (3.5,1.03-6) to [out=10, in=170] 
                            (4, 1-6) to [out=10,in=170] (4.5, 1.03-6) to [out=350, in=105] (5,-6);
  \draw[very thick,dashed] (6-0,1-6) to [out=170, in=10] (6-0.5,1.03-6) to [out=190, in=75] (6-1,0-6);

  %%%%%%%%%%%%%%%
  % time domain
  %% x
  \draw [<->, rounded corners, very thick](7.5,1.5) -- (7.5,0) -- (13.5,0) node[below] {$i$};
  \node[left] at (7.5,0.7) {$x$};
  \foreach \i in {1,...,22}{
    \draw[thick] (\i/4+7.5,0) -- (\i/4+7.5,{funcx(\i/6.)});
    \draw[fill] (\i/4+\talign,{funcx(\i/6.)}) circle[radius=0.05];}
  \draw[very thick, dashed, domain=0:22,samples=100] plot(\x/4+7.5, {funcx(\x/6.)});

  %% h \ast x
  \draw [<->, rounded corners, very thick](7.5,1.5-4) -- (7.5,0-4) -- (13.5,0-4) node[below] {$i$};
  \node[left] at (7.5,0.7-4) {$h\ast x$};
  \foreach \i in {1,...,22}{
    \draw[thick] (\i/4+7.5,-4) -- (\i/4+7.5,{funchx(\i/6.)-4});
    \draw[fill] (\i/4+\talign,{funchx(\i/6.)-4}) circle[radius=0.05];}
  \draw[very thick, dashed, domain=0:22,samples=50] plot(\x/4+7.5, {funchx(\x/6.)-4});

  %% y 
  \draw [<->, rounded corners, very thick](7.5,1.5-6) -- (7.5,0-6) -- (13.5,0-6) node[below] {$j$};
  \node[left] at (7.5,0.7-6) {$y$};
  \foreach \i in {1,...,22}
    \draw[gray] (\i/4+7.5,-6) -- (\i/4+7.5,{funchx(\i/6.)-6});
  \foreach \i in {0,...,7}{
    \draw[thick] (3*\i/4+7.5,-6) -- (3*\i/4+7.5,{funchx(3*\i/6.)-6});
    \draw[fill] (3*\i/4+\talign,{funchx(3*\i/6.)-6}) circle[radius=0.05];}
  \draw[very thick, dashed, domain=0:22,samples=50] plot(\x/4+7.5, {funchx(\x/6.)-6});


\end{tikzpicture}
\caption{Series of steps for down sampling shown in the frequency domain when $M=3$.}
\end{figure}
