\documentclass[11pt,preprint]{aastex}
\usepackage{natbib}
\bibliographystyle{apj}
\usepackage{amsmath,epstopdf}
\usepackage{tikz}
\usetikzlibrary{dsp,chains,shapes,arrows}
\usepackage{rotating,lscape,threeparttablex}
\DeclareGraphicsRule{.pdftex}{pdf}{*}{}

\DeclareMathAlphabet{\mathpzc}{OT1}{pzc}{m}{it}
\newcommand{\z}{\mathpzc{z}}
\newcommand{\SWARM}[1]{\texttt{#1\,SWARM}}
\newcommand{\APHIDS}[1]{\texttt{APHIDS}}
\newcommand{\falign}{0}
\newcommand{\talign}{7.5}

\begin{document}

\title{Resampling \SWARM{6/11} for VLBI \\ v0.1}

\author{The APHIDS Team\altaffilmark{1}}
\altaffiltext{1}{Harvard-Smithsonian Center for Astrophysics, 60 Garden Street, Cambridge, MA 02138, USA}

\begin{abstract}
ABSTRACT
\end{abstract}

\section{Background} \label{sec:background}
\subsection{VLBI}
\subsection{SDBE}

\section{Sample Rate Conversion} \label{sec:src_basics}

Sample rate conversion, or resampling, is the process of taking a signal, $x[i]$, sampled at a rate $f_0$ and 
calculating new samples $y[i]$ at a different sample rate, $f_1$.  For \APHIDS we are concerned with the case 
where the ratio between the two sample rates are rational, $f_1/f_0 = L/M$.  Operationally, resampling is the 
chained combination of upsampling and downsampling \citep{oppenheim10,lyons11}.  Upsampling, or interpolation, 
increases the sampling rate 
by an integer factor $L$ with the insertion of $L-1$ zeroes between the original $x[i]$.  A low-pass, 
anti-imaging filter smooths the signal and supresses the high frequency spectral images greater than the original 
$f_0$ that have been introduced by the zero inserts.  Figure XX sketches this process in the time and frequency 
domain for when $L$ equals 3. 

Downsampling, or decimation, by an integer factor of $M$ requires that the signal be first low-pass filtered to 
avoid aliasing 
at frequencies greater than the targeted $f_0/M$.  Only then can the sampling rate be reduced to $f_0/M$ by 
selecting every $M$th sample from the filtered signal.  Figure XX shows this process in the time and frequency 
domain for when $M$ equals 2.

Sample rate conversion by a rational factor $L/M$ is then a three step process where the original signal is 
expanded by $L$, filtered, and then decimated by $M$.  As illustrated by Figure \ref{fig:resample_basic}, 
the anti-imaging and 
anti-aliasing filters have been combined into a single filter with cutoff frequency of 
$1/\mathrm{max}(L,M) \times Lf_0/2$.  Under this scheme interpolation must precede decimation otherwise desired 
frequency components greater than $f_0/M$ cannot be preserved.

\begin{figure}[H!]
\centering
\label{fig:resample_basic}
\begin{tikzpicture}
   \node[dspnodeopen,dsp/label=left]  (c0) {$x[i]$};
   \node[dspsquare,right=of c0]                     (c1) {\upsamplertext{L}};
   \node[dspfilter,right=of c1]                     (c2) {$H(\z)$};
   \node[dspsquare,right=of c2]                    (c3) {\downsamplertext{M}};
   \node[dspnodeopen,right=of c3,dsp/label=right]  (c4) {$y[j]$};
   \foreach \i [evaluate = \i as \j using int(\i+1)] in {0,1,...,3}
       \draw[dspconn] (c\i) -- (c\j);
\end{tikzpicture}
\caption{Basic multirate resampling signal flow graph.}
\end{figure}

\iffalse % to save time don't compile sampling examples
 %% upsampling cartoon
\begin{figure}[t]
\begin{tikzpicture}[
declare function={ 
    funcx(\x) = -0.5*sin(2*pi*0.2*\x r) + 0.2*cos(2*pi*0.32*\x r) + 
                0.3*sin((2*pi*0.9*\x - 0.3) r) + 0.6*sin(2*pi*0.5*\x r);},
]
% frequency domain

%% |X|
\draw [<->, rounded corners, very thick](\falign,1.5) -- (\falign,0) --(\falign+6.5,0) node[below] {$f$};
\node[left] at (\falign,0.7) {$|X|$};
\node[below] at (\falign+1,0) {$f_0/2$};
\foreach \i in {1,...,6}
  \draw[thick] (\i, 0) -- (\i,0.1);
  \draw[very thick] (\falign,1) to [out=0, in=170] 
                    (0.5+\falign,1.03) to [out=350, in=105] 
                    (\falign+1,0);
  \draw[very thick,dashed] (\falign+1,0) to [out=75, in=190]  
                           (\falign+1.5,1.03) to [out=10, in=180] 
                           (\falign+2, 1) to [out=0,in=170] 
                            (\falign+2.5, 1.03) to [out=350, in=105] 
                            (\falign+3,0);
  \draw[very thick,dashed] (\falign+3,0) to [out=75, in=190]  
                            (\falign+3.5,1.03) to [out=10, in=180] 
                            (\falign+4, 1) to [out=0,in=170] 
                            (\falign+4.5, 1.03) to [out=350, in=105] 
                            (\falign+5,0);
  \draw[very thick,dashed] (\falign+6-0,1) to [out=170, in=0] 
                            (\falign+6-0.5,1.03) to [out=190, in=75] 
                            (\falign+6-1,0);

  %% W
  \draw [<->, rounded corners, very thick](\falign,-0.5) -- (\falign,-2) -- (\falign+6.5,-2);
  \node[left] at (\falign,0.7-2) {$|X_L|$};
  \node[below] at (\falign+3,-2) {$Lf_0/2$};
  \foreach \i in {1,...,6}
    \draw[thick] (\i, -2) -- (\i,0.1-2);
  \draw[very thick] (\falign,1-2) to [out=0, in=170] 
                    (0.5,1.03-2) to [out=350, in=105] 
                    (\falign+1,-2);
  \draw[very thick,] (\falign+1,-2) to [out=75, in=190]  
                     (\falign+1.5,1.03-2) to [out=10, in=180] 
                     (\falign+2, 1-2) to [out=0,in=170] 
                     (\falign+2.5, 1.03-2) to [out=350, in=105] 
                     (\falign+3,-2);
  \draw[very thick,dashed] (\falign+3,-2) to [out=75, in=190]  
                            (\falign+3.5,1.03-2) to [out=10, in=180] 
                            (\falign+4, 1-2) to [out=0,in=170] 
                            (\falign+4.5, 1.03-2) to [out=350, in=105] 
                            (\falign+5,-2);
  \draw[very thick,dashed] (\falign+6-0,1-2) to [out=170, in=0] 
                            (\falign+6-0.5,1.03-2) to [out=190, in=75] 
                            (\falign+6-1,-2);

%% H
  \draw [<->, rounded corners, very thick](\falign,-2.5) -- (\falign,-4) -- (\falign+6.5,-4);
  \foreach \i in {1,...,6}
    \draw[thick] (\i, -4) -- (\i,0.1-4);
  \node[left] at (\falign,0.7-4) {$|H|$};
  \node[below] at (\falign+1,-4) {$f_0/2$};
  \node[below] at (\falign+6,-4) {$Lf_0$};
  \draw[very thick] (\falign+0,1-4) -- (\falign+0.8,1-4) -- (\falign+1,0-4);
  \draw[very thick,dashed] (\falign+5,0-4) -- (\falign+5+0.2,1-4) -- (\falign+6,1-4);

  %% Y = H \ast X_L 
  \draw [<->, rounded corners, very thick](\falign,-4.5) -- (\falign,-6) -- (\falign+6.5,-6) node[below] {$f$};
  \foreach \i in {1,...,6}
    \draw[thick] (\i, -6) -- (\i,0.1-6);
  \node[left,align=right] at (\falign,0.7-6) {$|Y|$};
  \node[below] at (\falign+3,-6) {$f_1/2$};
  \draw[very thick] (\falign,1-6) to [out=0, in=170] 
                    (0.5,1.03-6) to [out=350, in=105] 
                    (\falign+1,-6);
  \draw[very thick,dashed] (\falign+6-0,1-6) to [out=170, in=0] 
                            (\falign+6-0.5,1.03-6) to [out=190, in=75] 
                            (\falign+6-1,-6);

  %%%%%%%%%%%%%%%
  % time domain
  %% x
  \draw [<->, rounded corners, very thick](\talign,1.5) -- (\talign,0) -- (\talign+6.5,0) node[below] {$i$};
  \node[left] at (\talign,0.7) {$x$};
  \foreach \i in {0,...,7}
    \draw[thick] (3*\i/4+\talign,0) -- (3*\i/4+\talign,{funcx(\i/2.)});
  \foreach \i in {0,...,7}
    \draw[fill] (3*\i/4+\talign,{funcx(\i/2.)}) circle[radius=0.05];
  \draw[very thick, dashed, domain=0:23,samples=50] plot(\x/4+\talign, {funcx(\x/6.)});

  %% x_L
  \draw [<->, rounded corners, very thick](\talign,-0.5) -- (\talign,-2) -- (\talign+6.5,-2) node[below] {$i$};
  \node[left] at (\talign,0.7-2) {$x_L$};
  \foreach \i in {0,...,7}{
    \draw[thick] (3*\i/4+\talign,-2) -- (3*\i/4+\talign,{funcx(\i/2.)-2});
    \draw[fill] (3*\i/4+\talign,{funcx(\i/2.)-2}) circle[radius=0.05];}
  \foreach \i in {0,...,7}
    \foreach \j in {1,2} 
    \draw[fill] (3*\i/4+\talign+\j/4,0-2) circle[radius=0.05];
  \draw[very thick, dashed, domain=0:23,samples=50] plot(\x/4+\talign, {funcx(\x/6.)-2});

 %% y
  \draw [<->, rounded corners, very thick](\talign,-4.5) -- (\talign,-6) -- (\talign+6.5,-6) node[below] {$j$};
  \node[left] at (\talign,0.7-6) {$y$};
  \foreach \i in {0,...,22}{
    \draw[thick] (\i/4+\talign,-6) -- (\i/4+\talign,{funcx(\i/6.)-6});
    \draw[fill] (\i/4+\talign,{funcx(\i/6.)-6}) circle[radius=0.05];}
  \draw[very thick, dashed, domain=0:23,samples=50] plot(\x/4+\talign, {funcx(\x/6.)-6});

\end{tikzpicture}
\caption{A sketch of upsampling when $L = 3$. The left column shows how the spectrum changes with each operation
and the right column mirrors the effect in the time domain.}
\label{fig:upsampling}
\end{figure}

 %% cartoon explaining down-sampling
 % http://tex.stackexchange.com/questions/124878/declare-function-for-tikzpicture
\begin{figure}
\begin{tikzpicture}[
declare function={ 
    funcx(\x) = -0.1*sin(2*pi*2.8*\x r) + 0.2*sin(2*pi*1.3*\x r) + 
                0.3*sin((2*pi*0.9*\x - 0.3) r) + 0.6*sin(2*pi*0.5*\x r);
    funchx(\x) = 0.3*sin((2*pi*0.9*\x - 0.3) r) + 0.6*sin(2*pi*0.5*\x r);},
]
  % frequency domain 

  %% |X(f)|
  \draw [<->, rounded corners, very thick](0,1.5) -- (0,0) --(6.5,0) node[below] {$f$};
  \node[left] at (0,0.7) {$|X|$};
  \foreach \i in {1,...,6}
    \draw[thick] (\i, 0) -- (\i,0.1);
  \node[below] at (3,0) {$f_0/2$};
  \draw[very thick] (0,1) to [out=10, in=170] (0.5,1.03)
                        to [out=350, in=190] (2,0.95) 
                        to [out=10, in=170] (2.5,1)
                        to [out=-10, in=110] (3,0);
  \draw[very thick,dashed] (6-0,1) to 
                            [out=170, in=10] (6-0.5,1.03) to 
                            [out=190, in=350] (6-2,0.95) to 
                            [out=170, in=10] (6-2.5,1) to [out=190, in=70] (6-3,0);

  %% |H(f)|
  \draw [<->, rounded corners, very thick](0,-0.5) -- (0,-2) -- (6.5,-2);
  \node[left] at (0,0.7-2) {$|H|$};
  \foreach \i in {1,...,6}
    \draw[thick] (\i, -2) -- (\i,0.1-2);
  \node[below] at (1,0-2) {$f_0/2M$}; \node[below] at (6,0-2) {$f_0$};
  \draw[very thick] (0,1-2) -- (0.8,1-2) -- (1,0-2);
  \draw[very thick,dashed] (5,0-2) -- (5.2,1-2) -- (6,1-2);

  %% |W(f)|
  \draw [<->, rounded corners, very thick](0,-2.5) -- (0,-4) -- (6.5,-4);
  \node[left] at (0,0.7-4) {$|HX|$};
  \foreach \i in {1,...,6}
    \draw[thick] (\i, -4) -- (\i,0.1-4);
  \node[below] at (1,0-4) {$f_0/2M$}; \node[below] at (6,0-4) {$f_0$};
  \draw[very thick] (0,1-4) to [out=10, in=170] (0.5,1.03-4) to [out=350, in=105] (1,0-4);
  \draw[very thick,dashed] (6-0,1-4) to [out=170, in=10] (6-0.5,1.03-4) to [out=190, in=75] (6-1,0-4);

  %% Y(f)
  \draw [<->, rounded corners, very thick](0,-4.5) -- (0,-6) -- (6.5,-6) node[below] {$f$};
  \node[left] at (0,0.7-6) {$|Y|$};
  \foreach \i in {1,...,6}
    \draw[thick] (\i, -6) -- (\i,0.1-6);
  \node[below] at (1,0-6) {$f_1/2$}; 
  \draw[very thick] (0,1-6) to [out=10, in=170] (0.5,1.03-6) to [out=350, in=105] (1,0-6);
  \draw[very thick,dashed] (1,0-6) to [out=75, in=190]  (1.5,1.03-6) to [out=10, in=170] 
                            (2, 1-6) to [out=10,in=170] (2.5, 1.03-6) to [out=350, in=105] (3,-6);
  \draw[very thick,dashed] (3,0-6) to [out=75, in=190]  (3.5,1.03-6) to [out=10, in=170] 
                            (4, 1-6) to [out=10,in=170] (4.5, 1.03-6) to [out=350, in=105] (5,-6);
  \draw[very thick,dashed] (6-0,1-6) to [out=170, in=10] (6-0.5,1.03-6) to [out=190, in=75] (6-1,0-6);

  %%%%%%%%%%%%%%%
  % time domain
  %% x
  \draw [<->, rounded corners, very thick](7.5,1.5) -- (7.5,0) -- (13.5,0) node[below] {$i$};
  \node[left] at (7.5,0.7) {$x$};
  \foreach \i in {1,...,22}{
    \draw[thick] (\i/4+7.5,0) -- (\i/4+7.5,{funcx(\i/6.)});
    \draw[fill] (\i/4+\talign,{funcx(\i/6.)}) circle[radius=0.05];}
  \draw[very thick, dashed, domain=0:22,samples=100] plot(\x/4+7.5, {funcx(\x/6.)});

  %% h \ast x
  \draw [<->, rounded corners, very thick](7.5,1.5-4) -- (7.5,0-4) -- (13.5,0-4) node[below] {$i$};
  \node[left] at (7.5,0.7-4) {$h\ast x$};
  \foreach \i in {1,...,22}{
    \draw[thick] (\i/4+7.5,-4) -- (\i/4+7.5,{funchx(\i/6.)-4});
    \draw[fill] (\i/4+\talign,{funchx(\i/6.)-4}) circle[radius=0.05];}
  \draw[very thick, dashed, domain=0:22,samples=50] plot(\x/4+7.5, {funchx(\x/6.)-4});

  %% y 
  \draw [<->, rounded corners, very thick](7.5,1.5-6) -- (7.5,0-6) -- (13.5,0-6) node[below] {$j$};
  \node[left] at (7.5,0.7-6) {$y$};
  \foreach \i in {1,...,22}
    \draw[gray] (\i/4+7.5,-6) -- (\i/4+7.5,{funchx(\i/6.)-6});
  \foreach \i in {0,...,7}{
    \draw[thick] (3*\i/4+7.5,-6) -- (3*\i/4+7.5,{funchx(3*\i/6.)-6});
    \draw[fill] (3*\i/4+\talign,{funchx(3*\i/6.)-6}) circle[radius=0.05];}
  \draw[very thick, dashed, domain=0:22,samples=50] plot(\x/4+7.5, {funchx(\x/6.)-6});


\end{tikzpicture}
\caption{Series of steps for down sampling shown in the frequency domain when $M=3$.}
\end{figure}

 %% cartoon explaining rational factor resampling
\begin{figure}[scale=1.1]
\begin{tikzpicture}[
declare function={ 
    funcx(\x) = -0.5*sin(2*pi*0.2*\x r) + 0.2*cos(2*pi*0.32*\x r) + 
                0.3*sin((2*pi*0.9*\x - 0.3) r) + 0.6*sin(2*pi*0.5*\x r);},
]
% frequency domain

%% |X|
\draw [<->, rounded corners, very thick](\falign,1.5) -- (\falign,0) --(\falign+6.5,0) node[below] {$f$};
\node[left] at (\falign,0.7) {$|X|$};
\node[below] at (\falign+1,0) {$f_0/2$};
\foreach \i in {1,...,6}
  \draw[thick] (\i, 0) -- (\i,0.1);
  \draw[very thick] (\falign,1) to [out=0, in=170] 
                    (0.5+\falign,1.03) to [out=350, in=105] 
                    (\falign+1,0);
  \draw[very thick,dashed] (\falign+1,0) to [out=75, in=190]  
                           (\falign+1.5,1.03) to [out=10, in=180] 
                           (\falign+2, 1) to [out=0,in=170] 
                            (\falign+2.5, 1.03) to [out=350, in=105] 
                            (\falign+3,0);
  \draw[very thick,dashed] (\falign+3,0) to [out=75, in=190]  
                            (\falign+3.5,1.03) to [out=10, in=180] 
                            (\falign+4, 1) to [out=0,in=170] 
                            (\falign+4.5, 1.03) to [out=350, in=105] 
                            (\falign+5,0);
  \draw[very thick,dashed] (\falign+6-0,1) to [out=170, in=0] 
                            (\falign+6-0.5,1.03) to [out=190, in=75] 
                            (\falign+6-1,0);

  %% W
  \draw [<->, rounded corners, very thick](\falign,-0.5) -- (\falign,-2) -- (\falign+6.5,-2);
  \node[left] at (\falign,0.7-2) {$|X_L|$};
  \node[below] at (\falign+3,-2) {$Lf_0/2$};
  \foreach \i in {1,...,6}
    \draw[thick] (\i, -2) -- (\i,0.1-2);
  \draw[very thick] (\falign,1-2) to [out=0, in=170] 
                    (0.5,1.03-2) to [out=350, in=105] 
                    (\falign+1,-2);
  \draw[very thick,] (\falign+1,-2) to [out=75, in=190]  
                     (\falign+1.5,1.03-2) to [out=10, in=180] 
                     (\falign+2, 1-2) to [out=0,in=170] 
                     (\falign+2.5, 1.03-2) to [out=350, in=105] 
                     (\falign+3,-2);
  \draw[very thick,dashed] (\falign+3,-2) to [out=75, in=190]  
                            (\falign+3.5,1.03-2) to [out=10, in=180] 
                            (\falign+4, 1-2) to [out=0,in=170] 
                            (\falign+4.5, 1.03-2) to [out=350, in=105] 
                            (\falign+5,-2);
  \draw[very thick,dashed] (\falign+6-0,1-2) to [out=170, in=0] 
                            (\falign+6-0.5,1.03-2) to [out=190, in=75] 
                            (\falign+6-1,-2);

%% H
  \draw [<->, rounded corners, very thick](\falign,-2.5) -- (\falign,-4) -- (\falign+6.5,-4);
  \foreach \i in {1,...,6}
    \draw[thick] (\i, -4) -- (\i,0.1-4);
  \node[left] at (\falign,0.7-4) {$|H|$};
  \node[below] at (\falign+1,-4) {$f_0/2$};
  \node[below] at (\falign+6,-4) {$Lf_0$};
  \draw[very thick] (\falign+0,1-4) -- (\falign+0.8,1-4) -- (\falign+1,0-4);
  \draw[very thick,dashed] (\falign+5,0-4) -- (\falign+5+0.2,1-4) -- (\falign+6,1-4);

  %% H * X_L 
  \draw [<->, rounded corners, very thick](\falign,-4.5) -- (\falign,-6) -- (\falign+6.5,-6);
  \foreach \i in {1,...,6}
    \draw[thick] (\i, -6) -- (\i,0.1-6);
  \node[left,align=right] at (\falign,0.7-6) {$|HX_L|$};
  \node[below] at (\falign+6,-6) {$Lf_0$};
  \node[below] at (\falign+3,-6) {$Lf_0/2$};
  \draw[very thick] (\falign,1-6) to [out=0, in=170] 
                    (0.5,1.03-6) to [out=350, in=105] 
                    (\falign+1,-6);
  \draw[very thick,dashed] (\falign+6-0,1-6) to [out=170, in=0] 
                            (\falign+6-0.5,1.03-6) to [out=190, in=75] 
                            (\falign+6-1,-6);
  %% Y
  \draw [<->, rounded corners, very thick](\falign,-4.5) -- (\falign,-8) -- (\falign+6.5,-8) node[below] {$f$};
  \foreach \i in {1,...,6}
    \draw[thick] (\i, -8) -- (\i,0.1-8);
  \node[left,align=right] at (\falign,0.7-8) {$|Y|$};
  \node[below] at (\falign+3,-8) {$Lf_0/M$};
  \draw[very thick] (\falign,1-8) to [out=0, in=170] 
                    (0.5,1.03-8) to [out=350, in=105] 
                    (\falign+1,-8);
  \draw[very thick,dashed] (\falign+2,-8) to [out=75, in=190]  
                            (\falign+2.5,1.03-8) to [out=10, in=180] 
                            (\falign+3, 1-8) to [out=0,in=170] 
                            (\falign+3.5, 1.03-8) to [out=350, in=105] 
                            (\falign+4,-8);
  \draw[very thick,dashed] (\falign+6-0,1-8) to [out=170, in=0] 
                            (\falign+6-0.5,1.03-8) to [out=190, in=75] 
                            (\falign+6-1,-8);

  %%%%%%%%%%%%%%%
  % time domain
  %% x
  \draw [<->, rounded corners, very thick](\talign,1.5) -- (\talign,0) -- (\talign+6.5,0) node[below] {$i$};
  \node[left] at (\talign,0.7) {$x$};
  \foreach \i in {0,...,7}
    \draw[thick] (3*\i/4+\talign,0) -- (3*\i/4+\talign,{funcx(\i/2.)});
  \foreach \i in {0,...,7}
    \draw[fill] (3*\i/4+\talign,{funcx(\i/2.)}) circle[radius=0.05];
  \draw[very thick, dashed, domain=0:23,samples=50] plot(\x/4+\talign, {funcx(\x/6.)});

  %% x_L
  \draw [<->, rounded corners, very thick](\talign,-0.5) -- (\talign,-2) -- (\talign+6.5,-2) node[below] {$i$};
  \node[left] at (\talign,0.7-2) {$x_L$};
  \foreach \i in {0,...,7}{
    \draw[thick] (3*\i/4+\talign,-2) -- (3*\i/4+\talign,{funcx(\i/2.)-2});
    \draw[fill] (3*\i/4+\talign,{funcx(\i/2.)-2}) circle[radius=0.05];}
  \foreach \i in {0,...,7}
    \foreach \j in {1,2} 
    \draw[fill] (3*\i/4+\talign+\j/4,0-2) circle[radius=0.05];
  \draw[very thick, dashed, domain=0:23,samples=50] plot(\x/4+\talign, {funcx(\x/6.)-2});

 %% h \ast x_L
  \draw [<->, rounded corners, very thick](\talign,-4.5) -- (\talign,-6) -- (\talign+6.5,-6) node[below] {$i$};
  \node[left] at (\talign,0.7-6) {$h \ast x_L$};
  \foreach \i in {0,...,22}{
    \draw[thick] (\i/4+\talign,-6) -- (\i/4+\talign,{funcx(\i/6.)-6});
    \draw[fill] (\i/4+\talign,{funcx(\i/6.)-6}) circle[radius=0.05];}
  \draw[very thick, dashed, domain=0:23,samples=50] plot(\x/4+\talign, {funcx(\x/6.)-6});


 %% y 
  \draw [<->, rounded corners, very thick](\talign,-6.5) -- (\talign,-8) -- (\talign+6.5,-8) node[below] {$j$};
  \node[left] at (\talign,0.7-8) {$y$};
  \foreach \i in {0,...,22}{
    %\draw[thick] (\i/4+\talign,-8) -- (\i/4+\talign,{funcx(\i/6.)-8});
    \draw[gray] (\i/4+\talign,-8) -- (\i/4+7.5,{funcx(\i/6.)-8});}
  \foreach \i in {0,...,11}{
    \draw[thick] (2*\i/4+\talign,-8) -- (2*\i/4+\talign,{funcx(2*\i/6.)-8});
    \draw[fill] (2*\i/4+\talign,{funcx(2*\i/6.)-8}) circle[radius=0.05];}
  \draw[very thick, dashed, domain=0:23,samples=50] plot(\x/4+\talign, {funcx(\x/6.)-8});

\end{tikzpicture}
\caption{$L = 3$, $M = 2$}
\end{figure}

\fi %\iffalse

Computationally, the basic sample rate conversion scheme described in \S\ref{sec:src_basics} is highly 
inefficient.  The filtering is applied at the highest possible sample rate, $Lf_0$, on
a time-series that is $(L-1)/L$ zeros by fraction. Next, the decimator discards an $(M-1)/M$ fraction of the 
samples.  As a result, this algorithm requires a lot of memory and spends precious clock 
time with wasted math.  There are multiple techniques for improving performance including 
multistage resampling, folded filter structures, and polyphase representations \citep{oppenheim10,lyons11,
vaidyanathan93}.

%\subsection{FIR filtering using Polyphase Representations}
\subsection{Polyphase representions and a time-varying FIR filter structure}
Polyphase filters use the Noble identities to move the filtering to precede the expander
or follow the decimator.  However, there still appears some waste as the former scheme (operating at $f_1$) 
produces zero-valued samples and the latter (operating at $f_0$) will discard samples.  An efficient variant
interchanges the decimator and expander blocks by shuffling the delays and decomposing the original n-tap FIR 
filter into $L$ sub-filters with $n/L$ taps \citep{crochiere81}.  This algorithm operates at the lowest 
possible rate of $f_0/M$.

References include \citep{wang01}.

\subsection{Linear and Nearest-Neighbor Interpolation}
An operationally simple option for resampling is to use linear interpolation to estimate the values that lie in 
between the original samples:
\begin{equation}
y[i] = x[j] + (x[j+1] - x[j]) (if_0/f_1 - j)
\end{equation}
where $j = \mathrm{floor}(if_0/f_1)$.  This is equivalent to applying a ``tent'' FIR filter with $2L$ taps on the 
upsampled $Lf_0$ time series before decimation.  Similarly, a $2L$ tap box-car filter is equivalent to 
nearest-neighbor interpolation:
\begin{equation}
y[i] = x[\mathrm{round}(if_0/f_1)].
\end{equation}

Figure \ref{fig:windows} shows the frequency response for linear and nearest-neighbor interpolation when 
the resampling factor $L/M$ is $64/39$.  Both methods have side lobes above $-40$\,dB and will introduce a large 
slope in the passband (See \S XX).

\begin{figure}[H]
\epsscale{1.0}
\plotone{windows.eps}
\caption{Frequency response for FIR filters equivalent to linear and nearest interpolation when $L/M = 64/39$.
The orange region shows the first Nyquist zone of the target $L/Mf_0$ sample rate.  Spectral components at all 
other frequencies are aliased into this region.}
\label{fig:windows}
\end{figure}
 
\subsection{Resampling in the Fourier Domain}
In contrast to the previous methods, one can also implement a rational $L/M$ sample rate conversion in the 
Fourier domain \citep{gold69,yeh82}.  After accumulating $kM$ samples at a clock rate of $f_0$ 
(where $k=1,2,3\cdots$), the DFT 
returns spectral components spaced at $f_0/kM$.  If $f_1 > f_0$, the resampled spectrum is generated by inserting
$p$ zeros to match the new $f_1$ while maintaining the correct frequency components: 
\begin{equation}
\frac{f_1}{kM+p} = \frac{f_0}{kM}
\end{equation} 
Solving for $p = k(L - M)$.  If $f_0 < f_1$, the spectrum is instead trimmed by $p$ samples.  The time series
sampled at $f_1$ can then be constructed from the inverse DFT.  

This method is equivalent to sinc interpolation using an ideal low-pass filter and introduces no 
error if the signal is perfectly periodic over the $kM$ accumulated samples.  When $L>M$, other values besides 
zero can 
be used for padding the spectrum (such as repeating the value at $f_0/2$) with some impact on the resulting 
error \citep{fraser89,bi11}.  For post-processing SWARM, FFT resampling is a good fit because one can 
access arbitrary large chunks of the time series without accumulators in hardware and has well-behaved errors
that can be easily modeled.  However, the speed of the FFT suffers due to the down-sampling factor of 39 and 
zero-padding in the case that the $L\gg M$ may be costly for memory.

\subsubsection{Effects from short DFTs}

It is not strictly true that DFT resampling requires no filter design. Trimming or padding de-facto multiplies 
the spectrum by a boxcar window which is equivalent to convolving a normalized sinc function with a 
\emph{periodic summation} of the original series.
Consequenty, errors will be wrapped into both edges of the 
resampled signal.  The extent of this error 
depends only on the width of the sinc function which is set by the resampling factors $L$ and $M$.  
\begin{equation} \label{eq:sinc}
B\,\mathrm{sinc}(B t) \overset{\mathcal{F}}{\Longleftrightarrow} \begin{cases} 1 \quad |f| < B/2 \\ 0 \quad |f| > B/2 \end{cases}
\end{equation}
where $B = \mathrm{min}(f_0,L/Mf_0)$.  
Therefore, the fraction of samples affected is inversely proportional to the number of samples.  Regardless, the 
number of samples for which this makes a large difference is small and can be roughly treated as less than a 
1\% effect when $N > 100\,\mathrm{max}(L/M,1)$.  One option to mitigate this error to stitch together 
overlapping, resampled segments \citep{bi11}, requiring more computation and memory.  A second strategy is to 
multiply the original signal by some window function that tapers to zero at its edges \citep{fraser89}.

Another phenomenon that becomes increasingly important for small DFTs is aliasing from the initial end-point
discontinuities.  The process of selecting $kM$ samples from the continuous signal $x(t)$ is represented as 
multiplication by a box-car window or convolution with a sinc function in the frequency domain.  The sinc window
widens for narrower box-car windows, causing spectral leakage and aliasing.  The ultimate result is also a loss 
in correlation amplitude.

\section{APHIDS} \label{sec:aphids}

\subsection{Resampling Block}

%% Loss table : /home/krosenfe/resampling/losses.py
% losses for FIR filtering, interpolation schemes, and FFT
\begin{deluxetable}{l|ccc}
\tablecolumns{4}
\tablewidth{0pc}
\tablecaption{Correlation coefficient loss from resampling \label{tab:loss}}
\tablehead{\colhead{Method} & \colhead{64/39} & \colhead{32/39} & \colhead{128/125}}
\startdata
Nearest-Neighbor            & 13\% & 17\% & 13\% \\
Linear                      &  5\% &  7\% &  5\% \\
Hamming window ($16L$ taps) &  1\% &  2\% &  2\% \\
FFT ($N=M$)                 &  1\% &  1\% & 0.2\%
\enddata
\end{deluxetable}

\subsection{Bit Growth}

\subsection{Quantization}

\subsection{Future Development}

\tikzstyle{line} = [draw,-latex']
\tikzstyle{cpu} = [rectangle, draw, fill=blue!20,
    text width=5em, text centered, rounded corners, minimum height=4em]
\tikzstyle{gpu} = [rectangle, draw, fill=red!20,
    text width=5em, text centered, rounded corners, minimum height=4em]
\tikzstyle{resamp} = [rectangle, draw, minimum height=4.5em, align=center, minimum width=15em] 

\begin{sidewaysfigure}
\begin{center}
%\caption{Overview of APHIDS.}
\begin{tikzpicture} [node distance = 5cm, auto]
\node [cpu] (mark6_in) {Mark6 Reader};
\node[gpu,below of=mark6_in](reader){Reader};
\node[gpu,right of=reader](rIFFT){rIFFT};
\node[gpu,right of=rIFFT](resampler){Resampler};
\node[cpu,right of=resampler](VDIF_pkt){VDIF packet \\ \& header};
\node[cpu,below of=VDIF_pkt](mark6_out){Mark6 Writer};

\path[line] (mark6_in) -- node[align=center] {BENG @ 2496 GHz \\ quantized to 2-bits}(reader);
\path[line] (reader) -- node[align=center] {spectra of \\ 16k complex64} (rIFFT);
\path[line] (rIFFT) --  node[align=center] {time series of \\ 32k float32} (resampler);
\path[line] (resampler) --  node[align=center] {time series \\ @ 2048 GHz}(VDIF_pkt);
\path[line] (VDIF_pkt) -- node[align=center]{VDIF @ 4096 GHz \\ quantized to 2-bits}(mark6_out);
\end{tikzpicture}
\end{center}
\end{sidewaysfigure}

\begin{figure}
\centering
\begin{tikzpicture}[node distance = 2.2cm, auto]
\node[align=center](init){};
\node[resamp,below of=init](rfft){batched real FFT \\ dimension of $39 \times N$};
\node[resamp,below of=rfft](trim){Trim band and shift frequency};
\node[resamp,below of=trim](rifft){batched real IFFT \\ dimension of $32 \times N$};
\node[align=center,below of=rifft](final){};

\path[line](init) -- node[align=left]{ time series \\ @ 2496\,MHz} (rfft);
\path[line](rfft) -- (trim);
\path[line](trim) -- (rifft);
\path[line](rifft) -- node[align=left]{ time series \\ @ 2048\,MHz}(final);
\end{tikzpicture}
\label{fig:resampling_block}
\caption{The resampling block trims the SWARM guard bands and shifts the passband to DC as part of the 
resampling operation.}
\end{figure}

\begin{figure}[H]
\epsscale{1.0}
\plotone{SWARM_amp_spectra.eps}
% /home/krosenfe/sdbe/software/prototyping/cuda/SWARM_amp_spectra.eps
\caption{This figure shows both channels of the detrended amplitude spectral density of SWARM along with the 
R2DBE (SMA single-dish in gray).}
\label{fig:swarm_amp_spec}
\end{figure}

\section{Results}

\acknowledgments 
\clearpage

\bibliography{sdbe}
\end{document}
