 %% upsampling cartoon
\begin{figure}[t]
\begin{tikzpicture}[
declare function={ 
    funcx(\x) = -0.5*sin(2*pi*0.2*\x r) + 0.2*cos(2*pi*0.32*\x r) + 
                0.3*sin((2*pi*0.9*\x - 0.3) r) + 0.6*sin(2*pi*0.5*\x r);},
]
% frequency domain

%% |X|
\draw [<->, rounded corners, very thick](\falign,1.5) -- (\falign,0) --(\falign+6.5,0) node[below] {$f$};
\node[left] at (\falign,0.7) {$|X|$};
\node[below] at (\falign+1,0) {$f_0/2$};
\foreach \i in {1,...,6}
  \draw[thick] (\i, 0) -- (\i,0.1);
  \draw[very thick] (\falign,1) to [out=0, in=170] 
                    (0.5+\falign,1.03) to [out=350, in=105] 
                    (\falign+1,0);
  \draw[very thick,dashed] (\falign+1,0) to [out=75, in=190]  
                           (\falign+1.5,1.03) to [out=10, in=180] 
                           (\falign+2, 1) to [out=0,in=170] 
                            (\falign+2.5, 1.03) to [out=350, in=105] 
                            (\falign+3,0);
  \draw[very thick,dashed] (\falign+3,0) to [out=75, in=190]  
                            (\falign+3.5,1.03) to [out=10, in=180] 
                            (\falign+4, 1) to [out=0,in=170] 
                            (\falign+4.5, 1.03) to [out=350, in=105] 
                            (\falign+5,0);
  \draw[very thick,dashed] (\falign+6-0,1) to [out=170, in=0] 
                            (\falign+6-0.5,1.03) to [out=190, in=75] 
                            (\falign+6-1,0);

  %% W
  \draw [<->, rounded corners, very thick](\falign,-0.5) -- (\falign,-2) -- (\falign+6.5,-2);
  \node[left] at (\falign,0.7-2) {$|X_L|$};
  \node[below] at (\falign+3,-2) {$Lf_0/2$};
  \foreach \i in {1,...,6}
    \draw[thick] (\i, -2) -- (\i,0.1-2);
  \draw[very thick] (\falign,1-2) to [out=0, in=170] 
                    (0.5,1.03-2) to [out=350, in=105] 
                    (\falign+1,-2);
  \draw[very thick,] (\falign+1,-2) to [out=75, in=190]  
                     (\falign+1.5,1.03-2) to [out=10, in=180] 
                     (\falign+2, 1-2) to [out=0,in=170] 
                     (\falign+2.5, 1.03-2) to [out=350, in=105] 
                     (\falign+3,-2);
  \draw[very thick,dashed] (\falign+3,-2) to [out=75, in=190]  
                            (\falign+3.5,1.03-2) to [out=10, in=180] 
                            (\falign+4, 1-2) to [out=0,in=170] 
                            (\falign+4.5, 1.03-2) to [out=350, in=105] 
                            (\falign+5,-2);
  \draw[very thick,dashed] (\falign+6-0,1-2) to [out=170, in=0] 
                            (\falign+6-0.5,1.03-2) to [out=190, in=75] 
                            (\falign+6-1,-2);

%% H
  \draw [<->, rounded corners, very thick](\falign,-2.5) -- (\falign,-4) -- (\falign+6.5,-4);
  \foreach \i in {1,...,6}
    \draw[thick] (\i, -4) -- (\i,0.1-4);
  \node[left] at (\falign,0.7-4) {$|H|$};
  \node[below] at (\falign+1,-4) {$f_0/2$};
  \node[below] at (\falign+6,-4) {$Lf_0$};
  \draw[very thick] (\falign+0,1-4) -- (\falign+0.8,1-4) -- (\falign+1,0-4);
  \draw[very thick,dashed] (\falign+5,0-4) -- (\falign+5+0.2,1-4) -- (\falign+6,1-4);

  %% Y = H \ast X_L 
  \draw [<->, rounded corners, very thick](\falign,-4.5) -- (\falign,-6) -- (\falign+6.5,-6) node[below] {$f$};
  \foreach \i in {1,...,6}
    \draw[thick] (\i, -6) -- (\i,0.1-6);
  \node[left,align=right] at (\falign,0.7-6) {$|Y|$};
  \node[below] at (\falign+3,-6) {$f_1/2$};
  \draw[very thick] (\falign,1-6) to [out=0, in=170] 
                    (0.5,1.03-6) to [out=350, in=105] 
                    (\falign+1,-6);
  \draw[very thick,dashed] (\falign+6-0,1-6) to [out=170, in=0] 
                            (\falign+6-0.5,1.03-6) to [out=190, in=75] 
                            (\falign+6-1,-6);

  %%%%%%%%%%%%%%%
  % time domain
  %% x
  \draw [<->, rounded corners, very thick](\talign,1.5) -- (\talign,0) -- (\talign+6.5,0) node[below] {$i$};
  \node[left] at (\talign,0.7) {$x$};
  \foreach \i in {0,...,7}
    \draw[thick] (3*\i/4+\talign,0) -- (3*\i/4+\talign,{funcx(\i/2.)});
  \foreach \i in {0,...,7}
    \draw[fill] (3*\i/4+\talign,{funcx(\i/2.)}) circle[radius=0.05];
  \draw[very thick, dashed, domain=0:23,samples=50] plot(\x/4+\talign, {funcx(\x/6.)});

  %% x_L
  \draw [<->, rounded corners, very thick](\talign,-0.5) -- (\talign,-2) -- (\talign+6.5,-2) node[below] {$i$};
  \node[left] at (\talign,0.7-2) {$x_L$};
  \foreach \i in {0,...,7}{
    \draw[thick] (3*\i/4+\talign,-2) -- (3*\i/4+\talign,{funcx(\i/2.)-2});
    \draw[fill] (3*\i/4+\talign,{funcx(\i/2.)-2}) circle[radius=0.05];}
  \foreach \i in {0,...,7}
    \foreach \j in {1,2} 
    \draw[fill] (3*\i/4+\talign+\j/4,0-2) circle[radius=0.05];
  \draw[very thick, dashed, domain=0:23,samples=50] plot(\x/4+\talign, {funcx(\x/6.)-2});

 %% y
  \draw [<->, rounded corners, very thick](\talign,-4.5) -- (\talign,-6) -- (\talign+6.5,-6) node[below] {$j$};
  \node[left] at (\talign,0.7-6) {$y$};
  \foreach \i in {0,...,22}{
    \draw[thick] (\i/4+\talign,-6) -- (\i/4+\talign,{funcx(\i/6.)-6});
    \draw[fill] (\i/4+\talign,{funcx(\i/6.)-6}) circle[radius=0.05];}
  \draw[very thick, dashed, domain=0:23,samples=50] plot(\x/4+\talign, {funcx(\x/6.)-6});

\end{tikzpicture}
\caption{A sketch of upsampling when $L = 3$. The left column shows how the spectrum changes with each operation
and the right column mirrors the effect in the time domain.}
\label{fig:upsampling}
\end{figure}
